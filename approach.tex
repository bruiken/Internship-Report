\chapter{Approach}\label{ch:approach}
This chapter covers more details on the approach that was taken in this research.
We will explain the details of the implementation of the \emph{fix} operator
designed to use in aCaPulCO. We will first look at the separation of the 
\emph{simple} and \emph{complex} constraints, and how they are represented in
aCaPulCO. Following that, we look at the design of the operator itself.

\section{Complex constraints}
First, recall that both the simple and complex constraints are cross-tree
constraints. Since aCaPulCO currently only accepts simple constraints in its
CPCO generation, the complex constraints are simply filtered out of the input
model. After parsing the cross-tree constraints, the arbitrary boolean formulas
are converted to CNF notation. The two accepted types of constraints 
(\emph{requires} and \emph{excludes} relationships) then look as follows:

\begin{itemize}
    \item \textit{Requires relationship}: A \emph{requires} relation is when a
          feature requires another feature for it to be enabled. We can see this
          as a logical implication, where if feature $f$ \emph{requires} feature 
          $g$, we have \( f \Rightarrow g \). Since the CNF notation only allows
          for conjunctions of disjunction, this is transformed to \( \neg f \lor g \).
    \item \textit{Excludes relationship}: An \emph{excludes} relation is when
          enabling a feature means that some other feature cannot be enabled. This
          can also be represented using a logical implication, where if feature $f$
          excludes feature $g$, we have \( f \Rightarrow \neg g \). In CNF notation
          we get \( \neg f \lor \neg g \).
\end{itemize}

In the preprocessing step for an input feature model in aCaPulCO, complex
constraints are already filtered out of the constraints. To be able to separate
the two types of constraints, we will still save the complex constraints in a
separate model. Our new \emph{fix} operator can then use this separate model to
retrieve only the complex constraints.

Now that we can distinguish the complex from the simple constraints, we can
continue onto the actual implementation of the operator itself.

\section{Fix operator}
The new \emph{fix} operator is tasked with dealing with complex constraints.
It will be called during the creation of the initial population, and after
crossover and mutation happen during the run of the genetic algorithm. This
is the same strategy as we have seen in MODAGAME~\cite{pascual2015modagame}.
The input configuration of our fix operator is a configuration which we know
is correct with the non cross-tree constraints of the feature model, as well
as the simple constraints. The strategy for fixing the constraints for the
complex constraints is very similar to the strategy used in MODAGAME. Where
MODAGAME needed to fix both the tree constraints and cross-tree constraints,
we only need to fix the cross-tree constraints.

Recall that MODAGAME fixes the cross-tree constraints by looking at all of
the input constraints and doing the following:
\begin{enumerate}
    \item If the current constraint is valid with the input configuration,
          go to the next constraint.
    \item Now we have to fix the constraint, which we can do by enabling or
          disabling exactly one feature in the constraint. This is because the
          input constraint is always a disjunction of features.
    \item If we manage to fix the constraint using one activation or
          deactivation, restart the process since we might have invalidated a previous
          constraint.
    \item If we do not manage to fix the constraint, result in failure.
\end{enumerate}

In MODAGAME, the enabling or disabling of features was quite expensive as no
information about what needs to happen to other features is stored. We 
have a simple way to enable or disable features: we have CPCOs. If we want to
enable feature \emph{A} to fix some constraint, we can simply call the CPCO for
the activation of \emph{A}.

Let us look at an example where our input configuration is as follows:
\( \left\{ \neg A, B, C, \neg D \right\} \). Now imagine we have constraints:
\(c_1: A \lor \neg B \lor D \) and \(c_2: \neg A \lor C \). 
Currently, constraint $c_1$ is invalid and is fixable
in three separate ways: we can either activate feature \emph{A} or \emph{D}, or
we can deactivate feature \emph{B}. It would be problematic to activate feature
\emph{A} however, as activating it would break constraint $c_2$. In our operator,
we randomly apply a strategy, similar to MODAGAME. We do this to not always get
stuck in the same loops (e.g. trying to activate and deactivate \emph{A} to adhere
to constraints $c_1$ and $c_2$).

\section{Limitation of approach}
This approach works, and the algorithm for fixing the constraints could even be
improved to reach success states faster. Yet there is a problem, to be able to
fix any complex cross-tree constraint, we may need to activate or deactivate any
given feature. Thus far, aCaPulCO did not require all the rules to be generated
to get results. It turns out that aCaPulCO outperforms the other
state-of-the-art tools, all the while not having generated close to all of the
CPCOs. This means that for a given feature model with for example 10.000 features,
it may only generate activation and deactivation rules for 1\% of these features.
It is of course likely that cross-tree constraints cover many features of the other
99\% of the non-generated rules. Our fix operator cannot repair those constraints,
as it can only use the rules.

One possible way of preventing this from happening might be to fix the
configurations without using the generated rules. If we do that, we manually flip
the bit in the bitstring for the appropriate feature and move on. There are two
problems with this approach:
\begin{itemize}
    \item We cannot assure that we keep true to the non cross-tree constraints, as
          we do not have functionality for that (the CPCOs made sure of this before).
    \item The crossover and mutation operators work not on the bitstrings themselves,
          but by combining rules applied to both parents. This means that the
          configurations themselves are not that important during the run of the
          algorithm. If we were to manually make changes, these changes would virtually
          be removed in the next crossover process.
\end{itemize}

Especially the second point is important for us. It would mean that during the entire
run of the algorithm, the fixes for the complex constraints are irrelevant. In that
case, we could also apply the \emph{fix} operator as the last step of the algorithm.
This is also not a real option as the \emph{fix} operator might fail, but also
because we want our genetic algorithm to keep into account these constraints. We
would not learn anything about the complex constraints if we ignore them throughout
the process.
