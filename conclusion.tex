\chapter{Conclusion}\label{ch:conclusion}
The goal of this research was to create a hybrid approach to deal with
complex constraints parallel to the CPCOs in aCaPulCO. In the process of
creating this operator and doing the research, it became clear that this
approach does not work for aCaPulCO. This is because aCaPulCO, in its
current state, can work with an incomplete rule generation. The \emph{fix}
operator that we implemented cannot work with this incompleteness.

The conclusion that we \emph{can} draw from this research, is that feature
models are complex, and deal with any type of cross-tree constraint
makes analysis prohibitively expensive. We also see this cost in the
generation of test data, where it is difficult to generate suitable cross
tree constraints on the fly. 

The scope of this internship is mainly limited by time, and a lot of time
has been put into trying to make the operator (in its current form) work.
When the limitation was identified, there was not enough time to delve into
the generation of the CPCOs, to try to make it more efficient (the current
limitation of the generation is explosive memory usage). This means that
after the discovery of the limitation, the focus was put on trying to find
the actual limits of the \emph{fix} operator. The idea here was to generate
suitable feature models, using BeTTy. This also turned out to be a timesink 
that did not yield positive results, because of the problems mentioned in
the previous chapter.

To make aCaPulCO work with complex cross-tree constraints, there are multiple
ways to go in future work. It is possible to look into the current form of
the CPCO generation and try to solve the memory issues in it. It could also
be interesting to further look into other ways of repairing the cross-tree
constraints, without the need for all of the CPCO rules to be generated. 
