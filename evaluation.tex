\chapter{Evaluation}\label{ch:evaluation}
This chapter covers the evaluation of the implemented operator. The quantifiable
results are minimal, because of the problems of the approach mentioned previously.
This chapter first starts with some notes on the generation of test data, and the
difficulties that come with it. We then briefly give some results on the time
taken by our implemented \emph{fix} operator.

\section{(Generation of) Test Data}\label{sec:testdata}
After the discovery of the aforementioned problems with the implementation of the
\emph{fix} operator in the current version of aCaPulCO, it should be clear that
we cannot test the large feature models that were used to evaluate the tool without
the \emph{fix} operator. These large feature models such as \emph{Linux} and
\emph{Automotive} (which are common feature models used to benchmark tools) are
simply too large to be able to generate all the CPCOs. For the small feature models,
we already have several other tools that can handle them. This means that we
need to get ahold of more input data.

To be able to control certain parameters in feature model generation, we could use
a tool such as BeTTy~\cite{segura2012betty} (BEnchmarking and TesTing on the analYsis
of feature models). It is a highly configurable tool that can be used to generate
(extended) feature models, it is easy to use parallel to aCaPulCO, as it is also
written in Java.

Because we want to benchmark specifically our \emph{fix} operator, we want to
generate feature models with many cross-tree constraints. Specifically, we want to
generate \emph{complex} cross-tree constraints. It should be clear that a feature
model with many cross-tree constraints would also need many features. If we have
a limited number of features, it is difficult to generate cross-tree constraints
that do not invalidate other constraints. We also do not want cross-tree constraints
to create dead features (features that can never be active), or false optional 
features (features that should be optional but are made mandatory because of
added constraints). For a tool such as BeTTy to create feature models that adhere
to these requirements, as well as a large number of cross-tree constraints, it needs
to generate large feature models with thousands of features. This brings us back to
our initial problem, where we did not want such large models, as it is not feasible
to generate all the CPCOs for them.

Apart from the difficulty of generating valid cross-tree constraints, it should
also be noted that a tool such as BeTTy can only generate simple constraints: it
is limited to creating \emph{requires} and \emph{excludes} relationships. For us,
this is not useful as aCaPulCO already can deal with these.
What we can do to prevent this, is to generate simple constraints and mark them
as complex constraints. With this approach, we do not solve the problem of the
generation that we mentioned previously.

\section{Results}\label{sec:results}
In the previous sections, we realised that our \emph{fix} operator is not a good
fit for the current version of aCaPulCO, as it requires all the CPCO rules to be
generated, while the other operators in aCaPulCO do not have this requirement.
The few tests that we have run on the new operator (with small feature models)
have not shown us any reasonable impact on the runtime of the genetic algorithm
(there was no measurable difference).

The smaller feature models that were benchmarked in the previous work on aCaPulCO
do not have to be benchmarked again with the \emph{fix} operator enabled, as these
small feature models do not contain any complex cross-tree constraints at all. That
would mean that our new operator does not get any work at all.
