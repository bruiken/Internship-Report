Software Product Lines (SPL) are a means of creating closely related software
products by storing all of the possible features in one codebase with an
accompanying feature model that tells how the features may interact with one
another~\cite{apel2016spl, kang1990fms}. These feature models thus contain the
information needed to create \emph{valid} products. We may want to disallow
features such as \emph{Windows} and \emph{Linux} to exist at the same time, for
example. Apart from these standard relations described in feature models, we
can also have \emph{extended feature models}~\cite{benavides2005extfms}. These
feature models also contain non-functional information for features, one can
imagine that we want to add information such as performance or cost. These
non-functional parameters can later be used for optimisation problems. Because
of the complexity of such models, combined with the fact that the multiple
objectives are orthogonal, this problem is 
NP-complete~\cite{ochoa2018npcomplete}. 

Solving this optimisation problem for feature models has been studied for more
than fifteen years and the most recent scalable solutions are all based on
meta-heuristic algorithms, specifically the genetic algorithm
IBEA~\cite{horcas2022breakit, zitzler2004ibea}. In these genetic algorithms,
activations of features are represented using bitstrings and standard mutation
and crossover operators are applied~\cite{ochoa2018npcomplete}. If one uses the
standard genetic operators, the represented selection of features is often
invalid, since it does not take the limitations of the feature model into
account~\cite{henard2015satibea, pascual2015modagame}. This requires validity
checking of solutions after applying the operators, and possibly also requires
some sort of ``fix'' operator to repair solutions. This in-situ fixing takes
time and thus hurts the performance of those algorithms. Work by Horcas et al.
mitigates this issue by creating specific genetic operators which do not break
the validity of feature models by design~\cite{horcas2022breakit}. The
operators are called \emph{consistency-preserving configuration operators}
(CPCOs). These operators only have to be generated once for a feature model,
which can be efficiently done before the actual genetic algorithm is
applied~\cite{horcas2022breakit}. These CPCOs outperform the previous in-situ
fixing algorithms. There is, however, a problem with CPCOs: the generation of
the operators is problematic for complex cross-tree constraints. Cross-tree 
constraints can typically be any boolean formula to define relations between
features. CPCO generation can deal with simple activations and deactivations,
but do not work well with general boolean formulas.

The goal of this internship is to apply a suggestion that already has been made by 
Horcas et al., where we might be able to mix the two above-discussed 
approaches. We could create CPCOs for the general case, where we have basic 
constraints and SATIBEA's~\cite{henard2015satibea} \emph{fix} operator for the 
remaining complex constraints. This might result in a situation where we have 
to do very little repairing because the CPCOs prevent most unwanted feature 
activations. This sounds like a best-of-both-words situation where we do not 
have to go through the explosive process of creating CPCOs from complex 
constraints, but we also do not have to apply the costly \emph{fix} operator 
many times. This idea requires firstly changes in the code of aCaPulCO, the 
tool created in~\cite{horcas2022breakit}. Secondly, we would like to apply the
same evaluation from this work again, this time also using the newly created 
variant of the algorithm.

We are going to see that while these goals seem fair and reasonable, they were
much harder to reach than initially thought. Trying to solve various
problems was the main time sink. Solving problems is a natural part of doing
research, but in this internship, the problems turned out to be unsolvable (at
least in the given timeframe of the internship.)

The style of this document is mainly that of a research paper, describing
background information, the research process, and -- in limited form -- the
results. But we will also see that this work contains my personal experiences
with the process as a whole. Personally, I feel as if those experiences make up
a large part of the entire internship, mainly because of the problems
encountered throughout it. The outline of this work is as follows:
we start with more background, first looking into Software Product Lines and
(extended) feature models, then also looking at aCaPulCO~\cite{horcas2022breakit}
in Chapter~\ref{ch:background}.
We will then look at complex constraints and how to deal with them in
Chapter~\ref{ch:approach}. Next, we delve into the evaluation of how we can
generate test data as well as give the limited results in
Chapter~\ref{ch:evaluation} Finally, we conclude this work in 
Chapter~\ref{ch:conclusion}.